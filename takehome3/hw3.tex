\documentclass[11pt]{article}
\usepackage{listings}
\usepackage[letterpaper]{geometry}
\usepackage{float}
\usepackage{graphicx}
\graphicspath{{./images/}}

\begin{document}

Morgan Rosenkranz 
EECE5644 
6/11/21 

\section*{Question 1:}

I started this question by defining the points of a cube and then assigning two of them as the means of each class.
I then gave each class two covariences which were just the identity matrix multiplied by a number between $0.4$ and $0.9$.
From there, I generated datasets of the different sample sizes given ($100,200,500,1000,2000,5000$), as well as a dataset of $100000$ samples.

%\begin{figure}[H]
%	\centering
%	\includegraphics[width=0.75\textwidth]{}
%	\caption{}
%\end{figure}

\section*{Question 2:}

%\begin{figure}[H]
%	\centering
%	\includegraphics[width=0.75\textwidth]{}
%	\caption{}
%\end{figure}

\section*{Code}
\subsection*{Code for Question 1}
\lstinputlisting[language=python]{q1.py}

\subsection*{Code for Question 2}
\lstinputlisting[language=python]{q2.py}
\end{document}
